% SIAM Shared Information Template
% This is information that is shared between the main document and any
% supplement. If no supplement is required, then this information can
% be included directly in the main document.


% Packages and macros go here
\usepackage{lipsum}
\usepackage{amsfonts}
\usepackage{graphicx}
\usepackage{epstopdf}
\usepackage{algorithmic}
\ifpdf
  \DeclareGraphicsExtensions{.eps,.pdf,.png,.jpg}
\else
  \DeclareGraphicsExtensions{.eps}
\fi

% Add a serial/Oxford comma by default.
\newcommand{\creflastconjunction}{, and~}

% Used for creating new theorem and remark environments
\newsiamremark{remark}{Remark}
\newsiamremark{hypothesis}{Hypothesis}
\crefname{hypothesis}{Hypothesis}{Hypotheses}
\newsiamthm{claim}{Claim}

% Sets running headers as well as PDF title and authors
\headers{Application of semi-local sa to approximate pattern matching}{N. Mishin, and D. Berezun}

% Title. If the supplement option is on, then "Supplementary Material"
% is automatically inserted before the title.
\title{Application of semi-local sa to approximate pattern matching
  % \thanks{Submitted to the editors DATE.
  % \funding{This work was funded by the Fog Research Institute under contract no.~FRI-454.}}
}

% Authors: full names plus addresses.
\author{Nikita Mishin\thanks{Saint Petersburg State University, Russia
    (\email{mishinnikitam@gmail.com}).}
\and Daniil Berezun\thanks{IntelliJ Labs Co. Ltd., Saint Petersburg, Russia
  (\email{daniil.berezun@jetbrains.com}).}}

\usepackage{amsopn}
\DeclareMathOperator{\diag}{diag}


%%% Local Variables: 
%%% mode:latex
%%% TeX-master: "ex_article"
%%% End: 
