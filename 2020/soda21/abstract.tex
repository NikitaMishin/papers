% REQUIRED
\begin{abstract}
  In the paper we study an application of semi-local sequence alignment (sa) algorithms
  to approximate pattern matching problem.
  We both developed two new algorithms as well as
  improved the existing near duplicate search algorithm (Programming and Computer Software'19).
  The key idea behind the algorithms is a usage of the underlying algebraic structure of semi-local sa (Tiskin, 2007)
  together with a novel data structure for submatrix maximum queries in Monge matrices (TALG'20).
  We also show that the improved near duplicate search algorithm not only has a better complexity
  but also preserves all declared properties.
  We show that the presented algorithms running time and space complexity are
  $O(max(|t||p|,\frac{|t| \log^2 |t|}{\log \log |t|} ))$ and
  $O(|t|)$  
  for the first one and
  $O(max(|t||p|,|t| \log |t| ))$ and $O(|t| \log |t|)$ for the last two, respectively,
  where $t$ is a text, $p$ --- pattern, and $v=O(1)$ is denominator of normalized mismatch score
  for semi-local sequence alignment.
  %Additionally, we show that sticky braid multiplication based semi-local lcs algorithm~cite{}
  %does not perform well in practrice beacause of its complex recursive structure. 
\end{abstract}
