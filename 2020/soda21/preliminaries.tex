\section{Preliminaries}
\label{section:preliminaries}

\subsection{Approximate pattern matching}
The approximate pattern matching problem ($AMatch$) defined as follows.
Given text $t$, pattern $p$ and some threshold $h$ the \emph{approximate pattern matching} problem ask for all substrings from text $t$ that have similarity score  with given pattern $p$ at least $h$ according to some similarity function $g$.  

There exist different kinds of extensions and particular cases of this problem.
For example, \emph{complete approximate pattern matching} (\emph{CompleteMatch}) that ask for
substrings of text $t$ that are exact clones of pattern $p$.
The approach for this special case of \emph{AMatch} is usage of well-know  algorithms such as  Aho-Korasic, BouerMurr, Knuth-Morris-Pratt, and so on.
The latter one have optimal running time complexity of $O(|p|+|t|)$ for \emph{CompleteMatch} problem\cite{}.
\emph{Approximate pattern matching with k mismatches} is an another example of special case \cite{}.
The search of \emph{pattern with wildcard symbols} or search set of patterns in text $t$\cite{}, multidimensional \emph{AMatch}\cite{} , search with lenght constraint of detected duplicates \cite{} are examples of such extension.
There exits many more examples of constrainsts, extensions and special cases of \emph{AMatch} problem\cite{}.


The one of the common approach to solve approximate pattern matching is the usage of solution of  string  similarity problem.
Latter represent a set of fundamental problems such as \emph{edit distance}, \emph{longest common subsequence}, \emph{sequence alignment}.
In this paper we primarily focuses on the last two. 


% as function $g$.
%%The one of the common approach is usage  
%to solve  is using \emph{sequence alignment} that measure similaruty







Describe approximate matching formally

\subsection{Semi-local lcs}
Describe semi-local lcs (definition), algorithms that solves (steady and and braid reducing)

\subsection{Monge matrix}
Describe monge property  


Say about range queries (about soda12, soda14 and new result that we will be used)

\subsection{Near-duplicate detection algorithm}

Describe luciv algo

