\section{Algorithm for near duplicate detection}
\label{section:luciv}
\red{TODO: third phase also may be imporved/ removed. By filtering in place.
Also prserves property.}

We now describe an improved version of Luciv et.al. algorithm \cite{luciv2019interactive} by utilizing a \emph{semi-local sa} solution.
Then we present proof that improved version preserves completnesess property.
It is achieved by imitating all phases of the algorithm. 
 

\subsection{Algorithm description}

\red{NEWALGO}.
Algorithm comprises of two phases.

At phase one (Line 1) semi-local sa problem is solved for the pattern $p$ against whole text $t$.

At phase two the search of largest text fragment within every window of size $L_{w} =\frac{|p|}{k}$ is performed.
Also the filtering of duplicate and intersected elements is perfomed in-place during this phase (this corresponds to third phase of \cite{luciv2019interactive}).
Note that length for text fragments (clones) is bounded by $|p|*k...\frac{|p|}{k}$ interval.
In terms of string-subsrtng matrix sliding window refers to square submatrix of size $L_{w}$.
Moreoveor, sliding window with step 1 refers to sliding square windows that goes diagonally with one step.

need piciutre.

Thus, it means that diagonal in string-substrin matrix of width $|p|*k..\frac{|p|}{k}$ and lenght $|t|$ corresponds to diagonal in string-substring matrix.
This means that it is sufficiently to visit only $O(\frac{|p|}{k} - |p|*k) * O(|t|) = O(|p||t|)$  cells to gain requiered text fragments.
Now we describe how to visit this cells.



\red{END NEW ALGO}


The algorithm comprises three phases as in \cite{luciv2019interactive}.
At phase one (Line 1) semi-local sa problem is solved for the pattern $p$ against whole text $t$.
This solution provides access to the string-substring matrix which allows performing fast queries of sa score for pattern $p$ against every substring of text $t$.

At the second phase text $t$ is scanning with a sliding window of length $L_{w}$ with step 1.
First, it checks that given substring $w$ that of a maximum possible size of $L_{w}$ have score that is higher or equal to a given threshold (Line 4).
If no, then this interval will not further be proceeded (Line 5) else this interval will be processed as follows.
First, for each prefix of text $t$ it finds suffix that has the highest alignment score with the maximal length among all suffixes with that score. 
It corresponds to the searching row position for each column in string-substring matrix with associated alignment score. 
Second, among these suffixes, one is selected with the highest score.
If several suffixes have the same score the one with maximal length is selected (Line 8).
Then if selected suffix has score higher than the threshold, then it is added to set $W_2$.

The third phase is the same as in \cite{luciv2019interactive}. 
More precisely, on the third phrase, set $W_{2}$ is filtered out in a such way that only non-intersected intervals are left.
It is simply the sorting of set $W_{3}$ by starts of intervals with following one way passage with filtering.  


\begin{algorithm}[H]
\caption{PATTERN BASED NEAR DUPLICATE
SEARCH ALGORITHM VIA SEMI-LOCAL SA}
\label{alg:patternMathing1}
Input: pattern $p$, text $t$, similiarity measure $k \in  [ \frac{1}{\sqrt{3}} ,1  ]$\\
Output: Set of non-intersected clones of pattern $p$ in text $t$
\begin{equation}
    k_{di}=|p|*(\frac{1}{k}+1)(1-k^2)
\end{equation}
\begin{equation}
 L_{w} = \frac{|p|} {k}
\end{equation}
Comment: $w_{i},w_{j}$ --- start and end positions of $w$ in text $t$ \\
Pseudocode:
\begin{algorithmic}[1]
\STATE{$W = semilocalsa(p,t)$}
\STATE{$W_2 = \emptyset$}
\FOR{$w \in t,  |w| = L_{w}$}
   \IF{ $W.stringSubstring(w_{i},w_{j}) < -k_{di}$}
   %\IF{ $sa(p,w) \geq -k_{di}$}
   \STATE{ \emph{continue}}
   \ENDIF
   \STATE{ $maximums = FindMaxForColumnsBySmawk(w)$}
   \STATE{ $max = FindMaxWithLenghtConstraint(maximums)$}
   \IF{$max \geq -k_{di}$}
   %\COMMENT{In There there was min thus here is minus} 
   \STATE{ add substring associated with max to $W_{2}$} 
   \ENDIF
\ENDFOR
\STATE{ $W_3 = UNIQUE(W_2)$}
\COMMENT{3rd phase unchanged}
\FOR{$w \in W_3$}
\IF{$\exists w^{'} \in W_3:w \subset w^{'} $}
\STATE{ $remove$ $w$ $from$ $W_3$}
\ENDIF

\ENDFOR
\RETURN $W_3$

\end{algorithmic}
\end{algorithm}


\begin{theorem}
Algorithm \ref{alg:patternMathing1} could  be solved in $ O(|p| * |t| )$ where $p$ is pattern, $t$ is text.

\red{Proof it}
\begin{displaymath}
    D = \diag(d_1,\dots,d_n)
  \end{displaymath}
\end{theorem}

\begin{theorem}
Algorithm \ref{alg:patternMathing1} preserves completnesses property of algorithm \cite{luciv2019interactive}.

\red{Proof it}
\begin{displaymath}
    D = \diag(d_1,\dots,d_n)
  \end{displaymath}
\end{theorem}
