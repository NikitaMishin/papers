\section{Introduction}
% approximate string matching: many areas
Approximate pattern matching is an important task in many fields such as computational biology, signal processing, text retrieval and etc.
It also refers to a duplicate detection subtask.
% basic formulation of approximate matching
In general form it formulates as follows: Given some pattern $p$ and text $t$ it asks to find all occurrences of pattern $p$ in text $t$ with some degree of similarity.

% short overview what constarins could be
There exists a lot of algorithms that solve the above problem.
Nonetheless, the number of suitable algorithms sharply decreases when the algorithm needs to meet some specific requirements imposed by running time, space complexity or specific criterion for the algorithm itself.
For example, recently there was developed an approach for interactive duplicate detection for software documentation~\cite{luciv2019interactive}.
The core of this approach is an algorithm that detects approximate clones of a given pattern $p$ with a specified degree of similarity and length boundaries for detected clones.
The main advantage of the algorithm is that it meets a specific requirement of completeness.
Nonetheless, it has an unpleasant time complexity.



   
The common approach of algorithm for approximate detection utilizes mainly algorithm for solving the longest commons subsequence (\emph{LCS}) problem.

The longest common subsequence is a well-known fundamental problem in computer science that also has many applications of its own.
The major drawback of it that it shows only the global similarity for given input strings.
For many tasks, it's simply not enough.
The approximate matching is an example of it.

There exist generalazation for \emph{LCS} called \emph{semi-local LCS}~\cite{} which overcome this constraint. 
The effective theoretical solutions for this generalized problem found applications to various algorithmic problems \todo{such as bla bla add cited}.
For example, there has been developed algorithm for approximate matching in the grammar-compresed strings~\cite{.}.

% research question Q1
Although the algorithms for \emph{semi-local LCS} have good theoretical properties, there is unclear how they would behave in practice for a specific task and domain.
% research question Q2
To show the applicability of semi-local lcs on practice we developed several algorithms based mainly on it and the underlying algebraic structure.
As well as developing new algorithms we improve the existing algorithm for duplicate detection in software documentationfrom~\cite{luciv2019interactive}.
It should be noted that improvement preserves all properties of this algorithm.
All presented algorithms also supports length constraints on the resulting substrings.
Finally, we provided and proved running time and space complexity for all presented algorithms.

%% The paper is organized as follows.
%% Blablabla
%% \cref{sec:main}, our new algorithm is in \cref{sec:alg}, experimental
%% results are in \cref{sec:experiments}, and the conclusions follow in
%% \cref{sec:conclusions}.

