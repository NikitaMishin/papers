\section{Introduction}
% approximate string matching intro
The approximate pattern matching is a well-known and studied problem.
It seeks to find all substrings of text $t$ that are close enough in terms of selected similarity function (TODO: or metric) to given pattern $p$.
This problem arises in many fields.
For example,  it applied to computational biology, signal and image processing, text retrieval and etc.

% Application in different fields entails various versions of this problem  with some 

% various extension

There also exist several variants of this problem that have different objectives with various implicitly or explicitly defined constraints.

The length constraint of matching subsequences is an example of the latter one.
It used in computational bioinformatics when the search  of a particular DNA subsequence in
a large DNA sequence or DNA DATABASE is performed.
Consider the case where we are given two matching substrings where the first one has a slightly higher similar score but has a significantly smaller size compared to the second one.
There may be a situation when the application will yield only the first one substring although the second one may be more significant biologically.
Thus, a specific lower bound on matching substrings should be set.

Now consider the opposite case when the first one has a slightly smaller similar score but has a significantly smaller size compared to the second one.
There could be a situation when the first one is much more significant in biology terms than the second one but the application will yield the second one. 
Thus, a specific upper bound on matching substrings should be set.

Same constraint arises in clone detection task when too small matching duplicates may be irrelevant  (article, interjections, punctuation marks, common words) whereas too big duplicates could not contains pattern at all (simply contains a word of pattern).


The common approach of algorithm for approximate pattern matching utilizes  algorithm for solving the longest commons subsequence (\emph{LCS}) and sequence alignment (\emphs{SA})  problems.

The longest common subsequence is a well-known fundamental problem in computer science that also has many applications of its own.
The major drawback of it that it shows only the global similarity for given input strings.
For many tasks, it's simply not enough.
The approximate matching is an example of it.

There exist generalazation for \emph{LCS} called \emph{semi-local LCS}~\cite{tiskin2008semi} which can overcome this constraint. 
The effective theoretical solutions for this generalized problem found applications to various algorithmic problems~\cite{tiskin2009periodic,tiskin2006longest,tiskin2011towards}.

% research question Q1
Although the algorithms for \emph{semi-local LCS} have good theoretical properties, there is unclear how they would behave in practice for a specific task and domain.
More particularly, how it could be applied to approximate pattern matching with mentioned lower and upper bound on length constraint on matching subsequences.
